% -- extra packages
% \usepackage{eucal}
\usepackage{nicefrac}
\usepackage{mathtools}
\usepackage[italicdiff]{physics}

% -- for pandoc-theorem
\usepackage{amsthm}
\theoremstyle{plain}
\newtheorem{theorem}{Theorem}[section]

\theoremstyle{definition}
\newtheorem{definition}{Definition}[section]
\newtheorem{proposition}{Proposition}[section]
\newtheorem{claim}{Claim}[section]

\theoremstyle{remark}
\newtheorem{lemma}{Lemma}[section]
\newtheorem{note}{Note}[section]
\newtheorem{example}{Example}[section]
\newtheorem{assumption}{Assumption}[section]

% less indentation of subitems
\usepackage{enumitem}
\setlist[enumerate]{leftmargin=1em}
\setlist[itemize]{leftmargin=1em}

% cases looking weird
\makeatletter
\def\env@cases{%
  \let\@ifnextchar\new@ifnextchar
  \left\lbrace
  \def\arraystretch{1.2}%
  \array{l@{\quad}l@{}}% Formerly @{}l@{\quad}l@{}
}
\makeatother

% -- shorthands
\newcommand{\1}[1]{\mathbb{1}_{{#1}}\,}

% -- delimeters
\DeclarePairedDelimiter\ceil{\lceil}{\rceil}
\DeclarePairedDelimiter\floor{\lfloor}{\rfloor}

% -- symbols
\newcommand{\ind}{\mathbin{{\perp\!\!\!\perp}}}
\newcommand{\iid}{\text{iid}}
\newcommand{\toinp}{\to_P}
\newcommand{\toas}{\to_{\text{a.s.}}}
\newcommand{\weakto}{\Rightarrow}
\let\upto\nearrow
\let\nil\varnothing
\AtBeginDocument{
  % to do this after unicode-math has done its work
  \renewcommand{\setminus}{\mathbin{\backslash}}%
}

% -- bbs and cals
\newcommand{\bbR}{\mathbb{R}}
\newcommand{\bbZ}{\mathbb{Z}}
\newcommand{\calA}{\mathcal{A}}
\newcommand{\calB}{\mathcal{B}}
\newcommand{\calF}{\mathcal{F}}
\newcommand{\calG}{\mathcal{G}}
\newcommand{\calM}{\mathcal{M}}
\newcommand{\calX}{\mathcal{X}}
\newcommand{\calY}{\mathcal{Y}}
\newcommand{\bX}{\mathbf{X}}

% -- distributions
\let\dist\thicksim
\newcommand{\distiid}{\thicksim_{\text{iid}}}
\DeclareMathOperator{\Bin}{Binomial}
\DeclareMathOperator{\Po}{Poisson}

% -- common operators
\renewcommand{\var}{\operatorname{var}}
\newcommand{\cov}{\operatorname{cov}}
\newcommand{\cbrt}[1]{\sqrt[3]{#1}}


